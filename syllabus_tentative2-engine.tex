% Options for packages loaded elsewhere
% Options for packages loaded elsewhere
\PassOptionsToPackage{unicode}{hyperref}
\PassOptionsToPackage{hyphens}{url}
\PassOptionsToPackage{dvipsnames,svgnames,x11names}{xcolor}
%
\documentclass[
  letterpaper,
  DIV=11,
  numbers=noendperiod]{scrartcl}
\usepackage{xcolor}
\usepackage{amsmath,amssymb}
\setcounter{secnumdepth}{-\maxdimen} % remove section numbering
\usepackage{iftex}
\ifPDFTeX
  \usepackage[T1]{fontenc}
  \usepackage[utf8]{inputenc}
  \usepackage{textcomp} % provide euro and other symbols
\else % if luatex or xetex
  \usepackage{unicode-math} % this also loads fontspec
  \defaultfontfeatures{Scale=MatchLowercase}
  \defaultfontfeatures[\rmfamily]{Ligatures=TeX,Scale=1}
\fi
\usepackage{lmodern}
\ifPDFTeX\else
  % xetex/luatex font selection
\fi
% Use upquote if available, for straight quotes in verbatim environments
\IfFileExists{upquote.sty}{\usepackage{upquote}}{}
\IfFileExists{microtype.sty}{% use microtype if available
  \usepackage[]{microtype}
  \UseMicrotypeSet[protrusion]{basicmath} % disable protrusion for tt fonts
}{}
\makeatletter
\@ifundefined{KOMAClassName}{% if non-KOMA class
  \IfFileExists{parskip.sty}{%
    \usepackage{parskip}
  }{% else
    \setlength{\parindent}{0pt}
    \setlength{\parskip}{6pt plus 2pt minus 1pt}}
}{% if KOMA class
  \KOMAoptions{parskip=half}}
\makeatother
% Make \paragraph and \subparagraph free-standing
\makeatletter
\ifx\paragraph\undefined\else
  \let\oldparagraph\paragraph
  \renewcommand{\paragraph}{
    \@ifstar
      \xxxParagraphStar
      \xxxParagraphNoStar
  }
  \newcommand{\xxxParagraphStar}[1]{\oldparagraph*{#1}\mbox{}}
  \newcommand{\xxxParagraphNoStar}[1]{\oldparagraph{#1}\mbox{}}
\fi
\ifx\subparagraph\undefined\else
  \let\oldsubparagraph\subparagraph
  \renewcommand{\subparagraph}{
    \@ifstar
      \xxxSubParagraphStar
      \xxxSubParagraphNoStar
  }
  \newcommand{\xxxSubParagraphStar}[1]{\oldsubparagraph*{#1}\mbox{}}
  \newcommand{\xxxSubParagraphNoStar}[1]{\oldsubparagraph{#1}\mbox{}}
\fi
\makeatother


\usepackage{longtable,booktabs,array}
\usepackage{calc} % for calculating minipage widths
% Correct order of tables after \paragraph or \subparagraph
\usepackage{etoolbox}
\makeatletter
\patchcmd\longtable{\par}{\if@noskipsec\mbox{}\fi\par}{}{}
\makeatother
% Allow footnotes in longtable head/foot
\IfFileExists{footnotehyper.sty}{\usepackage{footnotehyper}}{\usepackage{footnote}}
\makesavenoteenv{longtable}
\usepackage{graphicx}
\makeatletter
\newsavebox\pandoc@box
\newcommand*\pandocbounded[1]{% scales image to fit in text height/width
  \sbox\pandoc@box{#1}%
  \Gscale@div\@tempa{\textheight}{\dimexpr\ht\pandoc@box+\dp\pandoc@box\relax}%
  \Gscale@div\@tempb{\linewidth}{\wd\pandoc@box}%
  \ifdim\@tempb\p@<\@tempa\p@\let\@tempa\@tempb\fi% select the smaller of both
  \ifdim\@tempa\p@<\p@\scalebox{\@tempa}{\usebox\pandoc@box}%
  \else\usebox{\pandoc@box}%
  \fi%
}
% Set default figure placement to htbp
\def\fps@figure{htbp}
\makeatother





\setlength{\emergencystretch}{3em} % prevent overfull lines

\providecommand{\tightlist}{%
  \setlength{\itemsep}{0pt}\setlength{\parskip}{0pt}}



 


\KOMAoption{captions}{tableheading}
\makeatletter
\@ifpackageloaded{caption}{}{\usepackage{caption}}
\AtBeginDocument{%
\ifdefined\contentsname
  \renewcommand*\contentsname{Table of contents}
\else
  \newcommand\contentsname{Table of contents}
\fi
\ifdefined\listfigurename
  \renewcommand*\listfigurename{List of Figures}
\else
  \newcommand\listfigurename{List of Figures}
\fi
\ifdefined\listtablename
  \renewcommand*\listtablename{List of Tables}
\else
  \newcommand\listtablename{List of Tables}
\fi
\ifdefined\figurename
  \renewcommand*\figurename{Figure}
\else
  \newcommand\figurename{Figure}
\fi
\ifdefined\tablename
  \renewcommand*\tablename{Table}
\else
  \newcommand\tablename{Table}
\fi
}
\@ifpackageloaded{float}{}{\usepackage{float}}
\floatstyle{ruled}
\@ifundefined{c@chapter}{\newfloat{codelisting}{h}{lop}}{\newfloat{codelisting}{h}{lop}[chapter]}
\floatname{codelisting}{Listing}
\newcommand*\listoflistings{\listof{codelisting}{List of Listings}}
\makeatother
\makeatletter
\makeatother
\makeatletter
\@ifpackageloaded{caption}{}{\usepackage{caption}}
\@ifpackageloaded{subcaption}{}{\usepackage{subcaption}}
\makeatother
\usepackage{bookmark}
\IfFileExists{xurl.sty}{\usepackage{xurl}}{} % add URL line breaks if available
\urlstyle{same}
\hypersetup{
  pdftitle={HKU SOCI7016 - Quantitative research methods},
  pdfauthor={Kwan To Wong},
  colorlinks=true,
  linkcolor={blue},
  filecolor={Maroon},
  citecolor={Blue},
  urlcolor={Blue},
  pdfcreator={LaTeX via pandoc}}


\title{HKU SOCI7016 - Quantitative research methods}
\usepackage{etoolbox}
\makeatletter
\providecommand{\subtitle}[1]{% add subtitle to \maketitle
  \apptocmd{\@title}{\par {\large #1 \par}}{}{}
}
\makeatother
\subtitle{2025 Fall}
\author{Kwan To Wong}
\date{}
\begin{document}
\maketitle


\subsection{Course details}\label{course-details}

\begin{longtable}[]{@{}
  >{\raggedright\arraybackslash}p{(\linewidth - 6\tabcolsep) * \real{0.1294}}
  >{\raggedright\arraybackslash}p{(\linewidth - 6\tabcolsep) * \real{0.2706}}
  >{\raggedright\arraybackslash}p{(\linewidth - 6\tabcolsep) * \real{0.2235}}
  >{\raggedright\arraybackslash}p{(\linewidth - 6\tabcolsep) * \real{0.3765}}@{}}
\toprule\noalign{}
\begin{minipage}[b]{\linewidth}\raggedright
Format
\end{minipage} & \begin{minipage}[b]{\linewidth}\raggedright
Day
\end{minipage} & \begin{minipage}[b]{\linewidth}\raggedright
Time
\end{minipage} & \begin{minipage}[b]{\linewidth}\raggedright
Location
\end{minipage} \\
\midrule\noalign{}
\endhead
\bottomrule\noalign{}
\endlastfoot
Lectures & Wed (to be confirmed) & 7pm - 9:50pm & TBA \\
\end{longtable}

\subsection{Course objectives}\label{course-objectives}

In this course, you will learn the basics of quantitative methods as
applied to the social sciences. This involves two broad skill sets.
First, a good command of the conceptual foundations of statistics is
essential to understand why social scientists analyze data in some
specific ways. Second, a hands-on experience in the computing and
programming tools (R in our course) to both manage and analyze data is
crucial to be able to apply the conceptual foundations of statistics to
real-world data.

\subsection{Course learning outcomes}\label{course-learning-outcomes}

At the end of the course you should be able to:

\begin{itemize}
\tightlist
\item
  Use R for data analysis
\item
  Summarize and visualize data
\item
  Wrangle data into tidy forms
\item
  Understand the basic concepts of probability and statistics
\item
  Quantify uncertainty in data analysis
\item
  Be able to use regression models to analyze data
\end{itemize}

\subsection{Assessment Strategies
(tentative)}\label{assessment-strategies-tentative}

\begin{itemize}
\tightlist
\item
  Class participation (10\%)
\item
  \textbf{Midterm exam} (40\%)
\item
  Group Final project (50\%)

  \begin{itemize}
  \tightlist
  \item
    proposal (10\%)
  \item
    presentation (20\%)
  \item
    a blog/website (20\%)
  \end{itemize}
\end{itemize}

\subsection{Course schedule}\label{course-schedule}

\begin{longtable}[]{@{}
  >{\raggedright\arraybackslash}p{(\linewidth - 4\tabcolsep) * \real{0.0638}}
  >{\raggedright\arraybackslash}p{(\linewidth - 4\tabcolsep) * \real{0.1915}}
  >{\raggedright\arraybackslash}p{(\linewidth - 4\tabcolsep) * \real{0.7447}}@{}}
\toprule\noalign{}
\begin{minipage}[b]{\linewidth}\raggedright
Week
\end{minipage} & \begin{minipage}[b]{\linewidth}\raggedright
Date
\end{minipage} & \begin{minipage}[b]{\linewidth}\raggedright
Topic
\end{minipage} \\
\midrule\noalign{}
\endhead
\bottomrule\noalign{}
\endlastfoot
1 & 2025-09-03 & Overview and Introduction to R \\
2 & 2025-09-10 & Data visualization \\
3 & 2025-09-17 & Data wrangling \\
4 & 2025-09-24 & Probability and distributions \\
5 & 2025-10-01 & Sampling distributions \\
6 & 2025-10-08 & Confidence intervals and Hypothesis testing \\
7 & 2025-10-15 & Reading Week \\
8 & 2025-10-22 & Linear regression I \\
9 & 2025-10-29 & Linear regression II \\
10 & 2025-11-05 & Logistic regression and more \\
11 & 2025-11-12 & Fixed-effects and DID \\
12 & 2025-11-19 & \textbf{Midterm} \\
13 & 2025-11-26 & Final project presentations \\
\end{longtable}

\subsection{Midterm exam}\label{midterm-exam}

The midterm exam will be on \textbf{Nov 19th}. It will last 150 mins. It
will cover all the materials covered in the syllabus and it will be a
in-class exam. If you miss it, you will fail in this part. You will not
be allowed to collaborate with other students or be able to communicate
with any humans/AI/Internets about the exam. More information about the
exam will be provided as it approaches.

\subsection{Group Final project
presentation}\label{group-final-project-presentation}

Each group will have a group of 3 students, with plus/minus 1
flexibility. If you really want to do an individual project, please seek
my approval first.

This is a \texttt{data\ analysis} project about a particular topic
excites your group. No matter what the topic your group choose, your
group need formula at least one key research question, find data to
answer that question by using quantitative methods, and present those
results for educated population.

\begin{enumerate}
\def\labelenumi{\arabic{enumi}.}
\tightlist
\item
  A proposal (submission due: Nov 5th 5pm in Moodle)
\end{enumerate}

\begin{itemize}
\tightlist
\item
  write a short (one-page) proposal
\item
  state your research question
\item
  why this question is ``interesting''
\item
  formulate a hypothesis related to the research question
\item
  where you will get the data to test the hypothesis
\end{itemize}

\begin{enumerate}
\def\labelenumi{\arabic{enumi}.}
\setcounter{enumi}{1}
\tightlist
\item
  The final project presentation (\textbf{Nov 26th})
\end{enumerate}

It is an opportunity for you to present your final project to the class.
The presentation should include the following:

\begin{itemize}
\tightlist
\item
  An introduction to the research question
\item
  An overview of the data, including its source, collection method, and
  any relevant background information
\item
  An overview of the methods used
\item
  visualization of the results
\item
  A brief discussion of the implications of the research for theory,
  policy or practice
\item
  A conclusion
\item
  A brief discussion of the limitations of the study (optional)
\item
  A brief discussion of the future directions of the research (optional)
\end{itemize}

\begin{enumerate}
\def\labelenumi{\arabic{enumi}.}
\setcounter{enumi}{2}
\tightlist
\item
  A blog/website (\textbf{Nov 30th})
\end{enumerate}

\begin{itemize}
\item
  Present and interpret your findings in a blog not more than 800 words
  and attached with at least 3 visualization of data and results
\item
  published electronically online
  e.g.~\href{https://medium.com/}{Medium},
  \href{https://shorthand.com/}{Shorthand},
  \href{https://wordpress.com/}{WordPress},
  \href{https://github.blog/}{Github blog} or other online public
  platform.
\end{itemize}

\subsection{Textbooks and Reference}\label{textbooks-and-reference}

I will draw materials from the following books in this course:

\begin{itemize}
\item
  Imai, Kosuke and Nora Webb Willaims. 2022. Quantitative Social
  Science: An Introduction with Tidyverse, 2022. Princeton University
  Press.
\item
  Ismay, Chester and Albert Y. Kim. 2022. Statistical Inference via Data
  Science: A ModernDive into R and the Tidyverse.
\end{itemize}

\subsection{Late Policy}\label{late-policy}

\begin{itemize}
\item
  All course work assignment due dates will not be deferred except in
  cases of medical emergencies or family bereavement, for which written
  evidence must be supplied.
\item
  Applications for extension of submission deadline must be made via
  email to the course instructor \textbf{prior to the original
  deadline}.
\item
  Late submissions will be penalized \textbf{2 sub-grade} if submitted
  0-24 hours late and penalized \textbf{4 sub-grade} if submitted 24-48
  hours late. We will not accept any submissions after 48 hours --- we
  will get \textbf{F} for the task.
\item
  Technical difficulties are not acceptable reasons for lateness.
\end{itemize}

\subsection{Academic honesty and
plagiarism}\label{academic-honesty-and-plagiarism}

The university has high expectations for ethical conduct among students.
Students must always pursue academic honesty in their studies, as this
is a condition for authentic learning and creative intellectual
contributions. Committing the following (including but not limited to)
dishonest behaviour is unacceptable in any circumstances:

\begin{itemize}
\item
  Plagiarism (i.e.~use of other people's work without proper
  acknowledgement);
\item
  Submitting an assignment that is not the student's own work, including
  the use of prior students' work
\item
  Forgery of any document or certificate
\end{itemize}

Please be reminded that disciplinary actions in connection with the
violation of academic honesty may result in serious consequences.
Plagiarism may be handled by the individual teacher or reported to the
University Disciplinary Committee.

\begin{itemize}
\item
  Resources:

  \begin{itemize}
  \tightlist
  \item
    \href{https://tl.hku.hk/plagiarism/}{HKU Teaching \& Learning --
    Plagiarism}
  \item
    \href{https://learning.hku.hk/catalog/course/ilt01/}{HKU Online
    Learning -- Information Literacy Training: Academic Honesty}
  \end{itemize}
\item
  In this class, we will use AI tools, including ChatGPT, that harness
  large language models as pedagogical opportunities for learning and
  teaching in the course. Doing so will align with one of our course
  goals, which pertains to the evaluation of arguments and justification
  with evidence.
\item
  Our class agreement will follow the University's Policy on Use of
  Generative Artificial Intelligence for Teaching and Learning, which
  notes that ``students who fail to make a full declaration and proper
  citation according to the course requirements may face disciplinary
  action''.
\item
  For this class, all submitted work should be written in your own
  words. Just as you cannot take credit for others' writing in your
  assignments, you cannot use paraphrasing software (``spinbots'') or AI
  writing software (like ChatGPT) and submit the output as your own.
  Doing so in this is a violation of the University's
  \href{https://intraweb.hku.hk/reserved_1/tlearn/genai/gaitf-policy-dissemination-202309.pdf}{Policy
  on Use of Generative Artificial Intelligence for Teaching and
  Learning}, which notes that ``students who fail to make a full
  declaration and proper citation according to the course requirements
  may face disciplinary action''. At the beginning of the semester, we
  will identify examples of AI tools and discuss what constitutes
  plagiarism, cheating, and academic dishonesty. This will help to
  ensure that we are all on the same page about the policies for this
  course and how they connect to our learning outcomes.
\item
  Course privacy statement As noted in the Department's Guidelines on
  the recording of lectures and teaching sessions, students may not
  audio or video record class meetings without permission from the
  instructor (and guest speakers, when applicable). If the instructor
  grants permission or if the teaching team posts videos themselves,
  students may keep recordings only for personal use and may not post
  recordings on the Internet, or otherwise distribute them. These
  policies protect the privacy rights of instructors and students, and
  the intellectual property and other rights of the university. Students
  who need lectures recorded for the purposes of an academic
  accommodation should contact CEDARS Special Educational Needs (SEN)
  Support.
\end{itemize}

\subsection{Research ethics}\label{research-ethics}

This course involves completion of a research project as coursework. If
your research involves human participants, it is necessary to obtain
ethical approval prior to collecting data. Application forms should be
submitted to your instructor/tutor for endorsement and onward forwarding
to the Department for processing.

We are expected to adhere to the highest ethical standards in Social
Science Research. For example, the American Sociological Association's
\href{https://www.asanet.org/wp-content/uploads/asa_code_of_ethics-june2018a.pdf}{Code
of Ethics} sets forth the principles and ethical standards that underlie
sociologists' scientific and professional responsibilities and conduct.

\subsection{Student wellbeing}\label{student-wellbeing}

Life at university can get complicated. If you're feeling stressed,
overwhelmed, lost, anxious, depressed or are struggling with personal
issues, do not hesitate to contact CEDARS
\href{https://www.cedars.hku.hk/cope/cps}{Counselling and Psychological
Services (CoPE)}. These services are free and completely confidential.

+Room 301-323, 3/F, Main Building

\begin{itemize}
\item
  3917 8388
\item
  \href{mailto:cedars-cope@hku.hk}{\nolinkurl{cedars-cope@hku.hk}}
\item
  \href{https://www.cedars.hku.hk/cope/cps/appointment}{CoPE appointment
  booking}
\end{itemize}




\end{document}
